% Diese Zeile bitte -nicht- aendern.
\documentclass[course=erap]{aspdoc}

%%%%%%%%%%%%%%%%%%%%%%%%%%%%%%%%%
%% TODO: Ersetzen Sie in den folgenden Zeilen die entsprechenden -Texte-
%% mit den richtigen Werten.
\newcommand{\theGroup}{196} % Beispiel: 42
\newcommand{\theNumber}{IhreProjektnummer} % Beispiel: A123
\author{⁨Aleksandre Kandelaki \and Matthias Staritz \and Benjamin Liertz}
\date{Sommersemester 2020/21} % Beispiel: Wintersemester 2019/20
%%%%%%%%%%%%%%%%%%%%%%%%%%%%%%%%%

% Diese Zeile bitte -nicht- aendern.
\title{Gruppe \theGroup{} -- Abgabe zu Aufgabe \theNumber}

\begin{document}
\maketitle

\section{Einleitung}

\noindent\hspace*{15mm}%
Im Folgenden wird im Rahmen der Projektarbeit im Fach Einführung
in die Rechnerarchitektur an der TU München wird ein in linearer Algebra häufig 
benutztes Verfahren genauer beschrieben, implementiert und enstprechend dokumentiert.\\

\noindent\hspace*{15mm}%
Das Verfahren LU-Zerlegung, auch LR-Zerlegung genannt, bietet eine Interpretation des Gaußalgorithmus 
	 und somit eine Möglichkeit mit dem Computer lineare Gleichungssysteme zu 
	 lösen und Matrixinverse zu bestimmen.\\
 
\noindent\hspace*{15mm}%
Die LU-Zerlegung liefert mit Hilfe desselben Prinzips für jedes
   eindeutig lösbare Gleichungssystem, also wenn A regulär ist, zwei Dreiecksmatrizen L und 
   U und eine Pivot-Matrix P, wobei P*L*U = A ergibt.Hierbei haben die Matrizen besondere 
   Eigenschaften. L hat in allen Einträgen oberhalb der Diagonalen die Werte 0 und liefert 
   Auskunft über die Verwendeten Zeilenoperationen. U hingegen hat in allen Einträgen unterhalb
    der Diagonalen die Werte 0 und bildet somit eine Zeilenstufenform des LGS von der sich die 
    Lösungsmenge einfach ableiten lässt. Die Matrix P ist im eine Einheitsmatrix mit ggf. 
    vertauschten Zeilen mit der die Vertauschung von Zeilen nachvollziehbar wird.\\\\
    

\noindent\hspace*{15mm}%
Ein anschauliches Beispiel hierzu wäre das lineare Gleichungssystem:
\\\\\\\\\\\\\\\
\noindent\hspace*{15mm}%
Welches auch so dargestellt werden kann:  
\\\\\\\\\\\\\\\

\noindent\hspace*{15mm}%
Wird nun die LU-Zerlegung hierauf angewendet ergeben sich folgende Matrizen:
\\\\\\\\\\\\\\



\section{Lösungsansatz}


% TODO: Je nach Aufgabenstellung einen der Begriffe wählen
\section{Korrektheit/Genauigkeit}


\section{Performanzanalyse}


\section{Zusammenfassung und Ausblick}

% TODO: Fuegen Sie Ihre Quellen der Datei Ausarbeitung.bib hinzu
% Referenzieren Sie diese dann mit \cite{}.
% Beispiel: CR2 ist ein Register der x86-Architektur~\cite{intel2017man}.
\bibliographystyle{plain}
\bibliography{Ausarbeitung}{}

\end{document}

